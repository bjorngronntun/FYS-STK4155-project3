\documentclass{article}
\usepackage{amsmath}
\usepackage{amssymb}

\title{Project 3 on Machine Learning: Solving partial differential equations with neural networks}
\author{Bj{\o}rn Gr{\o}nntun}

\begin{document}
\maketitle

\begin{abstract}
    Neural networks have been shown to possess a general capability of mapping a given input to the correct output. We here show how this capability can be exploited in the context of two representative problems in applied mathematics: Solving partial differential equations and computing eigenvectors and eigenvalues of a real symmetric matrix. Comparisons with more traditional numerical methods are also performed.
\end{abstract}

\section{Introduction}
Although neural networks come in many versions, a quite common feature is that they process some kind of input (often represented by a vector in $\mathbb{R}^m$) which is then mapped to a output (often a vector in $\mathbb{R}^n$). If we have some way of measuring how well the output resembles the expected output, the weights of the network can be updated accordingly, thus yielding a better performance. Even though the most obvious application will be in a data science context, where our goal is to construct a predictive model from a set of measurements, neural network may also be used in other domains. Here, we will consider partial differential equations and eigenvectors of real symmetric matrices. It will be shown that both these problems can be thought of as function approximation problems - we seek a neural network that find solutions that gradually become closer to the true solution.

The examples we study will be rather simple - the one-dimensional heat equation with Dirichlet boundary conditions and finding the eigenvectors corresponding to the smallest and largets eigenvalue for relatively small matrices. However, it should be possible to use similar methods to solve more complicated problems. We have been able to obtain very good results for the eigenvector problem, whereas when it comes to solving PDEs, our work seems promising, but should be further improved.

As is often the case, this report only forms a small part of the entire research project. The corresponding code can be found in the Github repository \url{https://github.com/bjorngronntun/FYS-STK4155-project3}. The Python modules can be found in the \texttt{src} folder, whereas the Jupyter notebook where the results are obtained can be found in the \texttt{notebook} folder.

\section{Heat equation - physical interpretation and analytical solution}
If we have a rod of length $L=1$ with no lateral temperature loss, the temperature in the interior points of the rod can be modelled by a function $u(x, t)$, where $x\in (0, 1)$ is the distance from left point of the rod and $t$ is time. It is well established that the interior temperature will conform to the heat equation

\begin{equation}
    u_t = \alpha^2 u_{xx}
\end{equation}

For simplicity, we will let $\alpha=1$. As boundary condition, we will impose that the endpoints of the rod be kept at temperature 0. Initially, the temperature profile is given by $u(x, 0) = \sin(\pi x)$. Thus, the intitial-boundary value problem we are going to study is as follows:

\begin{equation}
    u_t = u_{xx}
\end{equation}
\begin{equation}
    u(0, t) = u(1, t) = 0
\end{equation}
\begin{equation}
    u(x, 0) = \sin(\pi x)
\end{equation}

The analytical solution to this problem can be found by the classical technique of separation by variables. Even if this derivation is given in every text on PDEs, we now proceed to give a quick sketch of this method. To start with, we look for solutions of the form $u(x, t) = X(x) T(t)$. This substitution transforms the PDE to $X(x) T'(t) = X''(x) T(t)$, which means that

\begin{equation}
    \frac{T'(t)}{T(t)} = \frac{X''(x)}{X(x)}
\end{equation}

As the left side of this equation depends solely on $t$ and the right side depends solely on $x$, the two sides must be equal to a common constant. Furthermore, this constant must be negative. (Explanation). This leads to the two ODEs

\begin{equation}
    T'(t) + \lambda^2 T(t) = 0\label{eq:1}
\end{equation}
and
\begin{equation}
    X''(x) + \lambda^2 X(x) = 0\label{eq:2}
\end{equation}

These equations are very easy to solve. \eqref{eq:1} has the general solution

\begin{equation}
    T(t) = C e^{-\lambda^2 t}
\end{equation}

whereas \eqref{eq:2} has the general solution

\begin{equation}
    X(x) = D \sin(\lambda x) + E \cos(\lambda x)
\end{equation}

Here, $C$, $D$ and $E$ are arbitrary constants. Thus the PDE is satisfied by all functions of the form

\begin{equation}
    u(x, t) = e^{-\lambda^2 t}[A \sin(\lambda x) + B \cos(\lambda x)]\label{eq:3}
\end{equation}

where $A$ and $B$ are arbitrary constants.

Now, we want to bring in the boundary conditions, the first of which is $u(0, t) = 0$. Putting $x=0$ in \eqref{eq:3}, we obtain $B e^{-\lambda^2 t} = 0$, so $B$ must be equal to 0. The second boundary condition, $u(1, t)$, now yields

\begin{equation}
    A e^{-\lambda^2 t} \sin(\lambda) = 0
\end{equation}

As can be seen, we must either have $A=0$, leading to the trivial solution $u\equiv 0$, or $\sin(\lambda) = 0$. Thus, $\lambda$ has to be an integer multiple of $\pi$. This leads to an entire family of solutions satisfying both the PDE and the boundary conditions. This family is given by

\begin{equation}
    u_n(x, t) = A_n e^{-(n \pi)^2 t} \sin(n \pi x)
\end{equation}

where $n \in \mathbb{Z}^{+}$.

It is easily verified that any linear combination of solutions satisfying the PDE and the boundary conditions will again be a solution. This leads us to consider solutions of the form

\begin{equation}
    u(x, t) = \sum_{n=1}^{\infty}{A_n e^{-(n \pi)^2 t} \sin(n \pi x)}
\end{equation}

In order that the solution satisfy the initial condition $u(x, 0) = \sin(\pi x)$, we put

\begin{equation}
    u(x, 0) = \sin(\pi x)
\end{equation}

that is

\begin{equation}
    \sum_{n=1}^{\infty}{A_n  sin(n \pi x)} = \sin(\pi x)
\end{equation}

In order to satisfy this equation, we simply let $A_1 = 1$ and $A_n = 0$ for $n \neq 1$. Now we finally have reached our analytical solution:

\begin{equation}
    u(x, t) = e^{-\pi^2 t} \sin(\pi x)
\end{equation}

It is easily verified that this solution satisfy the PDE, the boundary conditions and the initial condition. It can be shown that the solution is unique.

\section{Numerical methods - overview}
We seek a bivariate function, defined on $[0, 1] \times \mathbb{R}^+$. Of course, in a numerical setting we have to constrain the function domain somewhat: We choose to construct a grid of points on $[0, 1] \times [0, 1]$. If there are $n + 1$ grid lines parallell to the $t$ axis and $m + 1$ grid lines parallell to the $x$ axis, the distance between the grid lines when we move in the $x$ direction is $h = 1/n$ and the distance between the grid lines when we move in the $t$ direction is $k = 1/m$. The values for $x$ and $t$ will thus be given by

\begin{equation}
    x_j = jh
\end{equation}

where $j = 0, 1, ..., n$

and

\begin{equation}
    t_i = ik
\end{equation}

where $i = 0, 1, ..., m$.

When the grid has been properly defined, we can compute the values of our analytical solution on each point on the grid. As we develop our numerical solutions, they will be computed on the grid point as well. In order to measure how well the numerical solutions approach the analytical solution, we have chosen to opt for the maximal value of the absolute values of the differences between the numerical and the analytical solution. Of course, this only gives us information about the "worst case" approximation rather than the overall performance. In principle, a solution could be very good on most of the grid, but still bad according to this measure. Whether this error measurement is the correct one is certainly problem dependent, but we have chosen to use it due to its being very easy to compute.

\section{Finite difference scheme - theory}
In order to construct a finite-difference scheme to solve the heat equation, we use truncated forms of the Taylor series for $u$ to obtain the following approximations:

\begin{equation}
    u_t = \frac{1}{k}[u(x, t + k) - u(x, k)]
\end{equation}

and

\begin{equation}
    u_{xx} = \frac{1}{h^2}[u(x + h, t) - 2u(x, t) + u(x - h, t)]
\end{equation}

By using the notation $u_{i, j}$ for $u(x_j, t_i)$, we can write these equations as

\begin{equation}
    u_t = \frac{1}{k}[u_{i + 1, j} - u_{i, j}]
\end{equation}

and

\begin{equation}
    u_{xx} = \frac{1}{h^2}[u_{i, j+1} - 2u_{i, j} + u_{i, j-1}]
\end{equation}

The form of the heat equation we are studying demands that these be equal to each other. Solving for $u_{i+1, j}$, we obtain

\begin{equation}
    u_{i+1, j} = u_{i, j} + \frac{k}{h^2}[u_{i, j+1} - 2u_{i, j} + u_{i, j-1}]
\end{equation}

We now have a solution at each time step in terms of the solution at the previous time step. As we know the solution at $i=0$ (which corresponds to the initial condition), we now have a simple algorithm for moving forward in time and computing the solution for $i=1, ..., m$.


\section{Finite difference scheme - implementation and results}
The forward-difference method is implemented in \texttt{src/difference.py}. The testing is in \texttt{notebooks/solving-pde.ipynb}. In order to meet the stability criterion for this method, we begin by constructing a $11 \times 201$ grid, corresponding to step sizes of $\Delta x = 0.1$, $\Delta t = 0.005$. This just meets the stability criterion, since

\begin{equation}
    \frac{\Delta t}{\Delta x^2} = \frac{0.005}{0.01}=0.5
\end{equation}

In order to obtain even better approximations, we also construct a $101 \times 20001$ grid, corresponding to step sizes of $\Delta x = 0.01$, $\Delta t = 0.00005$. Again, this just meets the stability criterion, since

\begin{equation}
    \frac{\Delta t}{\Delta x^2} = \frac{0.00005}{0.0001}=0.5
\end{equation}

In order to evaluate how well our numerical solutions approach the analytical solution, we look at the maximal value of the absolute differences between the two. The relevant computations are in \texttt{notebooks/solving-pde.ipynb}. We find that when using the $11 \times 201$ grid, we obtain a maximal absolute difference of 0.006, whereas when using the $101 \times 20001$ grid, the maximal absolute difference is reduced to $6.05 \cdot 10^{-5}$.

\section{Neural networks}
We now turn to a completely different method for obtaining numerical solutions of differential equations. A neural network should in principle be able to approximate the values of a function on a given input grid. We just have to make sure that the solutions proposed by our method actually fulfill the initial conditions and boundary conditions for our problem. How this is done is somewhat problem dependent, but it is usually possible to formulate a trial solution of the type

\begin{equation}
    g_{\textrm{trial}}(\mathbf{x}, t) = h_1(\mathbf{x}, t) + h_2(\mathbf{x}, t, N(\mathbf{x}, t, P))
\end{equation}

Here, $h_1$ depends on the inputs $\mathbf{x}, t$ only, whereas $h_2$ is allowed to depend on the outputs of the outputs of the neural network. The state of the neural network is given by $P$, which represents its weights and biases. The role of $h_1$ and $h_2$ is to make sure any solution respect the initial and boundary conditions. In our case, we choose

\begin{equation}
    g_{\textrm{trial}}(x, t) = (1 - t)\sin(\pi x) + x(1 - x) t N(x, t, P)
\end{equation}

It is easily verified that $g_{\textrm{trial}}(x, 0) = \sin(\pi x)$ for any value of $x$, meaning that any solution will conform to our initial condition. Moreover, we see that $g_{\textrm{trial}}(0, t) = g_{\textrm{trial}}(1, t) = 0$ for any value of $t$, which means that our solutions always respect our boundary conditions.

We train our neural network by evaluating how well our trial solutions solve the PDE. As we want our solutions to solve

\begin{equation}
    u_t = u_{xx}
\end{equation}

that is

\begin{equation}
    u_t - u_{xx} = 0
\end{equation}

we use as our cost function $C = \textrm{MSE}((g_{\textrm{trial}})_t - (g_{\textrm{trial}})_{xx}, 0)$ and use this to train our network, thus obtaining gradually better approximations to the true solution.

Our Tensorflow implementation of this method is in \texttt{src/neural.py} and the results are obtained in \texttt{notebooks/solving-pde.ipynb}. For the $11 \times 201$ grid, we obtain quite good results with two hidden layers, each with twenty neurons and tanh activation functions. The maximal absolute difference between our solution and the analytical solution is 0.008, which is somewhat larger than the result obtained by the forward-difference method, but this could probably be improved further. For the $101 \times 20001$ grid our implemention became unacceptably slow, and we were not able to obtain any results.

\section{Comparison of methods}
As we have shown, the neural networks are able to obtain acceptable results in comparison with a more traditional forward-difference method. The neural network approach, however, is much slower. This may be alleviated in different ways - one could try running our implementation on faster machines or utilizing the GPU. It may also be an idea to try stochastic gradient descent in this case. The advantage of the neural network based method is perhaps not so much in speed as in ease of implementation. In our case, we were able to develop a finite-difference algorithm quite easily, but there may be cases where that is much more difficult, and in those cases the neural network based method should compete favourably.

\section{Eigenvalue problems}
Given a matrix $A \in \mathbb{R}^{n \times n}$, an \textit{eigenpair} $(\lambda, x), x \neq 0$ is a solution of the equation

\begin{equation}
    Ax = \lambda x
\end{equation}

Here, the scalar $\lambda$ called an \textit{eigenvalue} and $x \in \mathbb{R}^n$ is called an \textit{eigenvector}. Here, we only consider real symmetric matrices. It is easily proved that any eigenvalue of a real symmetric matrix is a real number. Finding eigenvectors and eigenvalues of such matrices is an important problem in engineering, and several numerical methods have been proposed for computing them. We here propose a very simple method employing neural networks. Given any non-zero vector $x \in \mathbb{R}^n$, the corresponding Rayleigh quotient for a real symmetric matrix $A$ is defined as

\begin{equation}
    R_A(x) = \frac{x^T A x}{x^T x}
\end{equation}

A trivial fact is that if $x$ is an eigenvector, its Rayleigh quotient will be equal to the corresponding eigenvalue. Moreover, it can be shown that the Rayleigh quotient is maximized and minimized by two of matrix' eigenvectors. This is quite easy to the show in the case where $A$ is symmetric, because then $A$ has a spectral decomposition $A=UDU^T$, where $U$ is orthogonal and $D$ is diagonal with the eigenvalues of $A$ on the diagonal. By putting $y=U^T x$ and noting that the condition $x \neq 0$ will be equivalent to $ y \neq 0$, since $U$ is non-singular, we get

\begin{align}
    \max_{x \neq 0}{R_A (x)}
    &=\max_{x \neq 0}{\frac{x^T A x}{x^T x}}\\
    &=\max_{x \neq 0}{\frac{x^T U D U^T x}{x^T x}}\\
    &=\max_{y \neq 0}{\frac{y^T D y}{y^T y}}\\
    &=\max_{y \neq 0}{ \frac{\sum_{i=1}^{n}{\lambda_i {y_i}^2 } } { \sum_{i=1}^{n} { {y_i}^2 } } } \\
    &\leq \frac{ \lambda_1  { \sum_{i=1}^{n} { {y_i}^2 } }} {  \sum_{i=1}^{n} { {y_i}^2 } }\\
    &=\lambda_1
\end{align}

So the Rayleigh quotient is bounded above by the largest eigenvalue $\lambda_1$. As this value is actually attained by the corresponding eigenvector, we conclude that this eigenvector maximizes $R_A (x)$. In the same way, we can show that the minimum value is $\lambda_n$, the smallest eigenvalue, which is attained by another eigenvector.

By maximizing or minimizing the Rayleigh quotient, we should therefore be able to find the eigenvectors corresponding to the largest and smallest eigenvalue. It is then a simple matter to find the eigenvalues using the Rayleigh quotient. The function \texttt{eigen} in \texttt{src/eigenvectors.py} works in the following way: A random vector is inputed into a dense neural network, yielding a proposed eigenvector as output. We compute the corresponding Rayleigh quotient and use gradient descent to update the weights of the network, thus reducing the Rayleigh quotient. This process is iterated until the reduction of the Rayleigh quotient in subsequent iterations gets below a certain tolerance, at which point we stop the process.

In the notebook \texttt{notebooks/finding-eigenvalues.ipynb} we use our implementation to find unit eigenvectors of a $6 \times 6$ real symmetric matrix. We find that a neural net with 5 input nodes and two hidden layers, each with 5 nodes and tanh activation function, perfoms quite well. Comparison with the eigenvectors found by \texttt{numpy.linalg.eig} yields a mean squared deviation of $1.057 \cdot 10^{-13}$ for one eigenvector and $1.33 \cdot10^{-11}$ for the other. In practice, this suggests our simple method achieve state-of-the-art perfomance on this kind of problem, at least when accuracy is concerned.

\section{Wrapping up}
In this paper, we have studied two classical problems in applied mathematics: Solving partial differential equations and computing eigenvectors (which may be used to find eigenvalues.) Based on the idea that neural networks are capable of approximating functions, we have tried to obtain solutions to these problems using neural networks. Our results show that this is possible, but that further work should be conducted in order to obtain more effective and general methods.

\begin{thebibliography}{3}
    \bibitem{geron}
    Aur\'elien G\'eron.
    \textit{Hands-On Machine Learning with Scikit-Learn and TensorFlow}
    O'Reilly, 2017

    \bibitem{introduction}
    Gareth James, Daniela Witten, Trevor Hastie, Robert Tibhirani.
    \textit{An Introduction to Statistical Learning}
    Springer, 2013

    \bibitem{elements}
    Kristine Baluka Hein.
    \textit{Data Analysis and Machine Learning: Using Neural networks to solve ODEs and PDEs}
    (\url{https://compphysics.github.io/MachineLearning/doc/pub/odenn/pdf/odenn-minted.pdf})

    \bibitem{elements}
    Zhang Yi, Yan Fu, Hua Jin Tang
    \textit{Neural Networks Based Approach for Computing Eigenvectors and Eigenvalues of Symmetric Matrix}
    (\url{https://www.sciencedirect.com/science/article/pii/S0898122104901101})
\end{thebibliography}
\end{document}
